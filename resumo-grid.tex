\documentclass{article}
\usepackage[utf8]{inputenc}
\usepackage[T1]{fontenc}
\usepackage{listings}
\usepackage{graphicx}
\usepackage{placeins}

\author{Renê Cardozo \\ 
        rene.cardozo@usp.br}

\title{Resumo de implementação dos cenários}

\date{}

\begin{document}
\maketitle

\section{Arranjo da Simulação}
Nós, representando veículos, foram criados em uma rede wifi Ad-Hoc, sendo posicionados em uma malha com espaçamento de 5
unidades, verticalmente e horizontalmente. O tamanho máximo das linhas desta malha foi calculado pela quantidade de nós dividida
por 5.

Em cada nó da rede foi utilizado o modelo de mobilidade ns3::RandomWalk2dMobilityModel com as configurações padrões do
ns-3.

Neste modelo o primeiro nó da conexão recebia pacotes de todos os outros nós    

\end{document}

