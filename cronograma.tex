\documentclass{article}
\usepackage{multirow}
\usepackage{graphicx}
\usepackage{placeins}
\usepackage[utf8]{inputenc}
\usepackage[T1]{fontenc}
\usepackage{tabularx}
\usepackage{url}
\usepackage{listings}

\renewcommand{\tablename}{Tab.}

\author{Renê Cardozo \\ 
        rene.cardozo@usp.br}

\title{Análise de Desempenho de Novos Protocolos da Camada de Transporte na Internet das Coisas \\
\begin{Large} Cronograma - MAC0215 \end{Large}
}

\date{}

\begin{document}
\maketitle

\section*{Cronograma}
A partir do dia 31/08, contam-se 14 semanas até o dia 11/12, quando será feita a entrega do relatório final da
disciplina. Dadas as 100 horas obrigatórias da disciplina, há a necessidade de dedicar em média 7 horas por semana para
a execução da pesquisa. Serão 28 horas para os primeiros três meses e 16 horas para Dezembro.

\FloatBarrier

\begin{itemize}
\item Setembro
    \begin{itemize}
        \item Estudo do simulador de redes ns-3 (16h)
        \begin{enumerate}
            \item Módulo de aplicações em sockets (8h)
            \item Módulo de redes sem fio (8h)
        \end{enumerate}
        \item Estudo da biblioteca de sockets de sistemas UNIX (8h)
    \end{itemize}
\item Outubro
    \begin{itemize}
        \item Estudo do módulo que permite a simulação do QUIC no ns-3 (leitura de documentação e experimentação) (10h)
        \item Estudo do módulo de energia do ns-3 (10h)
        \item Estudo de cenários de IoT simulados no ns-3 (8h)
    \end{itemize}
\item Novembro
    \begin{itemize}
        \item Integração entre os módulos do QUIC e de energia do ns-3 (6h)
        \item Realização de simulações em diferentes cenários e análise inicial dos resultados (10h)
        \item Estudo do módulo de mobilidade do ns-3 (12h)
    \end{itemize}
\item Dezembro
    \begin{itemize}
        \item Integração dos três módulos estudados. (6h)
        \item Realização dos experimentos finais e análise final dos resultados (10h)
    \end{itemize}
\end{itemize}

Para que seja possível acompanhar o cumprimento do cronograma, será utilizado um repositório no Github, no qual serão
colocados códigos de simulações e resumos de estudos realizados, bem como uma planilha que marcará quando uma atividade
for realizada.

\url{https://github.com/reneepc/MAC0215}

\url{https://docs.google.com/spreadsheets/d/1KiiLYd02EGBaGABIPCL55H3GX7m5ZoPnCu9HOxwmAKA/edit?usp=sharing}
\end{document}

