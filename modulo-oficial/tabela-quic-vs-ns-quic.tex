\documentclass{article}
\usepackage[utf8]{inputenc}
\usepackage{placeins}
\usepackage[T1]{fontenc}
\usepackage{listings}

\author{Renê Cardozo \\ 
        rene.cardozo@usp.br}

\title{Tabela QUIC vs ns-QUIC}

\date{}

\begin{document}
\maketitle

\begin{table}[h]
\centering

\begin{tabular}{|p{3cm}|p{5cm}|p{5cm}|}
    \hline
    Característica & QUIC* & nsQUIC \\
    \hline
    Versão & Utiliza a versão 16 do Draft da IETF** como base para o módulo, recebendo sua última atualização em junho
    de 2019. & Realiza a integração direta da implementação do proto-QUIC, o qual encontra-se descontinuado [5] desde 7
    de maio de 2019. A integração foi realizada com uma versão de 2017 do proto-QUIC. \\
    \hline
    Criptografia & Não implementa a stack do TLS, contudo é capaz de simular a negociação de pacotes para o
    estabelecimento de uma conexão segura. & Possui o código da camada da segurança herdado da implementação real,
    contudo este código foi suprimido para ser incorporado ao simulador. \\
    \hline
    Organização & Modular; Escrito nativamente para o simulador. & Adaptação de uma implementação já existente
    disponibilizada pela empresa desenvolvedora do protocolo. \\
    \hline
    Controle de Congestionamento & Devido a sua característica modular, podem ser utilizados algoritmos já
    implementados para o TCP em modo legado, o algoritmo padrão do QUIC na data do draft 16, bem como algoritmos
    desenvolvidos ou modificados pelo usuário. & Há apenas o
    algoritmo padrão implementado no proto-QUIC, não sendo possível alterá-lo com facilidade. \\
    \hline
    Documentação & Possui um artigo descrevendo as funções de cada classe, relações entre as mesmas, fluxo de dados
    e funcionamento dos métodos. & Possui um artigo que descreve sua incorporação ao simulador, bem como as
    características do protocolo no proto-QUIC. Porém, por se tratar de uma integração direta, não há uma descrição das
    funções das classes dentro do simulador. \\
    \hline
    Integração & Sua relação com outros módulos do ns-3 é mais clara, uma vez que é possível conhecer as classes que
    realizam a comunicação entre cada camada. & A relação entre as camadas não é claramente descrita. \\
    \hline
\end{tabular}
\caption{Diferenças entre o módulo QUIC [1,2], implementado nativamente no ns-3 e o módulo nsQUIC [3,4]}
\end{table}

\FloatBarrier

* Módulo nativo ao simulador disponível no repositório de aplicações do ns-3.
** Internet Engineering Task Force

\paragraph{Referências}

1. De Biasio, Alvise et al. “A QUIC Implementation for Ns-3.” Proceedings of the 2019 Workshop on ns-3 (2019): n. pag.
Crossref. Web. \\
2.ns-3 App Store, disponível em: apps.nsnam.org/app/quic/ \\
3. CAMARINHA, Diego de Araujo Martinez. Análise de desempenho do nsQUIC: um módulo para smulação do protocolo QUIC.
2018. Dissertação (Mestrado em Ciência da Computação) - Instituto de Matemática e Estatística, Universidad de São Paulo,
São Paulo, 2018. doi:10.11606/D.45.2018.tde-16102018-181616. Acesso em: 2020-08-06. \\
4. COUTO, Yan Soares; CAMARINHA, Diego; BATISTA, Daniel Macêdo. nsQUIC: uma extensao para simulação do protocolo QUIC no
NS-3. Anais.. Porto Alegre: SBC, 2018.Disponível em:
http://www.sbrc2018.ufscar.br/wp-content/uploads/2018/04/180520\_1.pdf \\
5. Repositório google/proto-quic no Github, disponível em: https://github.com/google/proto-quic
\end{document}

