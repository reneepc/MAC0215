\documentclass[14pt]{extarticle}
\usepackage[brazilian]{babel}
\usepackage[utf8]{inputenc}
\usepackage[T1]{fontenc}
\usepackage{listings}

\author{Renê Cardozo \\ 
        rene.cardozo@usp.br}

\title{Guia Rápido de Utilização do ns-3}

\date{}

\begin{document}
\maketitle

O primeiro passo para qualquer simulação no ns-3 é a definição de nós de conexão, armazenando-os em uma instância da
classe NodeContainer. Esta instância pode gerar nós automaticamente através da função Create(n\_nos) a qual cria
um número n de nós. Outra função possível é passar dois ou mais objetos NodeContainer para a função Create, gerando um
container concatenado destes.

Por serem objetos básicos, as instâncias da classe Node possuem características muito ilustrativas do modelo de software
adotado no simulador.

\section{Objetos}
Objetos no ns-3 podem ser criados tal qual em C++, contudo, para um melhor aproveitamento da memória e adição de
funcionalidades, o ns-3 implementa seus próprios métodos de criação de objectos.

Há três classes principais no ns-3, das quais derivam grande parte dos objetos encontrados no simulador:

\begin{itemize}
    \item Object
    \item ObjectBase
    \item SimpleRefCount
\end{itemize}

Apesar de não obrigatório, a herança destas classes dá propriedades especiais aos objetos. As classes que derivam de
Object possuem acesso ao sistema de atributos e tipos do ns-3, bem como o sisistema de agregação de objetos e contagem
de referência por smart-pointers.

\subsection{Contagem de Referência}

A contagem de referência é utilizada para facilitar o gerenciamento de memória. Todas as vezes que um ponteiro é
criado, uma chamada é realizada para a função Ref() e a contagem de referência do objeto é incrementada. O usuário deve
então aplicar a função Unref() ao ponteiro. Assim, quando a contagem de referência chegar a zero, o ponteiro pode ser
eliminado.

A classe Ptr<> age de maneira similar a classe intrusive\_ptr da biblioteca Boost, assumindo que a classe utilizada em
seu template implementa os métodos Ref() e Unref().

Para a criação dos nós a classe utiliza a função CreateObject<>(), a qual é tambem é central para o ns-3. Objetos derivados
da classe Object() nunca devem ser criados com a função new do C++. É necessário utilizar a função CreateObject para
realizar a contagem de referências de maneira correta.

Se o objeto não derivar de nenhum objeto do ns-3, mas é de desejo que suas referências sejam contadas, utiliza-se a
função Create<>(), como por exemplo, para a criação de pacotes.

\subsection{Sistema de Agregação}

No ns-2, muitos protocolos eram estendidos utilizando herança e polimorfismo, tal como diferentes versões do TCP. Para
tornar este tipo de utilização mais fácil, o ns-3 usa um design pattern que evita eventuais problemas.

A classe Node é um bom exemplo de utilização do sistema de agregação. Não existem classes derivadas da classe Node,
apesar de sua utilização em centenas de combinações de protocolos. Os protocolos devem ser agregados ao nó. Por exemplo:

\begin{lstlisting}
static void 
AddIpv4Stack(Ptr<Node> node) {
    Ptr<Ipv4L3Protocol> ipv4 = CreateObject<Ipv4L3Protocol>();
    ipv4->SetNode(node);
    node->AggregateObject(ipv4);
    Ptr<Ipv4Impl> ipv4Impl = CreateObject<Ipv4Impl>();
    ipv4Impl->SetIpv4(ipv4);
    node->AggregateObject(ipv4Impl);
}
\end{lstlisting}

Apenas um objeto de cada tipo pode ser agregado a uma instância.

As partes do protocolo são agregadas ao objeto. Assim, se uma aplicação necessitar da implementação deste protocolo, pode
requerir em runtime através do método GetObject<>();. Se o objeto não possuir o atributo especificado no template de
GetObject<>(); a função retornará null.

\subsection{Fábricas de Objetos}

Uma maneira comum de criar diversos objetos iguais é utilizando a classe ObjectFactory, a qual permite a configuração
prévia destes por meio de:

\begin{lstlisting}
void SetTypeId(TypeId tid);
void Set(std::string name, const AttributeValue &value);
Ptr<T> Create (void) const;
\end{lstlisting}

O primeiro método especifica o tipo do objeto criado, o segundo especifica atributos para os objetos criados, e o
terceiro permite a criações dos objetos em si. Por exemplo:

\begin{lstlisting}
ObjectFactory factory;
factory.SetTypeId("ns3::FriisPropagationLossModel");
factory.Set("SystemLoss", DoubleValue(2.0));
Ptr<Object> object = factory.Crate();
factory.Set("SystemLoss", DoubleValue(3.0));
Ptr<Object> object = factory.Create();
\end{lstlisting}

\section{Configuração}
No ns-3 temos dois principais aspectos de configuração: a topologia da simulação e como os objetos estão conectados; e
os valores utilizados pelos modelos nesta topologia.

Muitos objetos do ns-3 derivam da classe Object, a qual possui propriedades de organização do sistema e melhora no
gerenciamento de memória.

Dentro da classe são armazenados metadados sobre: a classe base, os construtores da classe, os atributos da classee se
estes são read-only ou read-write, e ainda os valores permitidos para cada atributo.

Objetos que herdam a classe Object deve ser alocados na heap utilizando o método CreateObject<>(). Alguns helpers podem
derivar de ObjectBase(), os quais são alocados na stack.

\subsection{TypeId}

As classes derivadas de Object podem possuir um TypeId que é utilizado para o sistema de agregação e gerenciamento de
componentes. Esse identificador é uma string única identificando a classe, sua classe base, o conjunto de construtores
da classe e uma lista dos atributos públicos desta.

O header da classe node inclui uma função estática GetTypeId(). Definida da forma:

\begin{lstlisting}
TypeId Node::GetTypeId(void) {
    statis TypeId tid = TypeId("ns3::Node")
    .SetParent<Object>()
    .SetGroupName("Network")
    .AddConstructor<Node>()
    .AddAttribute("DeviceList",
                "The list of devices associated to this Node.",
                ObjectVectorValue(),
                MakeObjectVectorAccessor(&Node::m\_devices),
                MakeObjectVectorChecker<NetDevice>())
    .AddAttribute("ApplicationList",
                "The list of applications associated to this Node.",
                ObjectVectorValue(),
                MakeObjectVectorAccessor(&Node::m\_applications),
                MakeObjectVectorChecker<Application>())
    .AddAttribute("Id",
                "The id (unique integer) of this Node.",
                TypeId:ATTR\_GET,
                UintegerValue(0),
                MakeUintegerAccessor(&Node::m\_id),
                MakeUintegerChecker<uint32\_t>());

    return tid;
}
\end{lstlisting}

A função SetParent<Object>() é utilizada para permitir casting e mecanismos de agragação. Também herda os atributos da
classe pai, Object.

O método AddConstructor<Node>() é utilizada é utilizado em conjunto com a fábrica de objetos, permitindo a construção de
objetos mesmo que o usuário não tenha conhecimeto exato sobre o construtor da classe.

Por fim, AddAttribute() associa uma string com um valor fortemente tipado da classe. Deve-se ainda informar uma string
de ajuda para a linha de comando. Cada atributo é associado com mecanismos para acessá-lo em uma variável da classe. O
Checker é utilizado para validar os valores em questão de tipos e de máximos e mínimos.

Pode-se criar nós através de:

\begin{lstlisting}
Ptr<Node> n = CreateObject<Node>();
\end{lstlisting}

ou, por meio de:

\begin{lstlisting}
ObjectFactory factory;
const std::string typeId = "ns3::Node";
factory.SetTypeId(typeId);
Ptr<Object> node = factory.Create<Object>();
\end{lstlisting}

No caso da inicialização por meio da fábrica de objetos, não há conhecimento da classe concreta do nó.

\subsection{Atributos}

No sistema de atributos do ns-3, se quisermos prover um valor padrão para o atributo de uma classe ou acessar o valor em
uma classe já instanciada, temos que consultar o objeto TypeId da classe.

A função AddAttribute() no método GetTypeId() de cada classe realiza diversas ações sobre um atributo. Liga sua variável
privada a uma string pública, indica o valor da variável, provê um texto de ajuda com o significado de seu valor,
atribui um "Checker" que será utilizado para avaliar a correta tipagem a intervalo da variável.

A especificação destas características pela função AddAttribute em GetTypeId torna os atributos seus valores padrão
acessíveis no espaço de nomes dos atributos, o qual é baseado em strings que definem cada um dos atributos.

\subsection{Argumentos da Linha de Comando}

Pode-se acrescentar a leitura de valores da linha de comando por meio da classe CommandLine e seu método AddValue, o
qual recebe uma string que indica o nome o atributo, uma mensagem de ajuda e a variável que armazenará o valor.

Após determinar quais valores devem ser lidos, utiliza-se o método Parse() para que as variáveis sejam devidamente
atualizadas.

\section{Topologia}
Após criar os nós para representar as máquinas, é necessário definir a topologia na qual eles estarão conectados.

Helpers são muito utilizados para definir os dispositivos que permitem um nós utilizar determinado tipo de tecnologia de
comunicação. Para isso, separamos os nós de interesse em um container separado, iniciamos o helper modificando os
devidos atributos e utilizamos este helper para instalar os dispositivos de rede nos nós. É de interesse guardar o
retorno da função de instalação na casse NetDeviceContainer.

\subsection{Wi-Fi}

No caso de redes wireless, geralmente os nós possuem diferentes funções na rede, como o de ponto de acesso e estações de
acesso. Assim, não é possível utilizar apenas um helper para configurar os dois dispositivos, mas um para definir o
canal de comunicação, o qual é associado a camada física da rede. Estes dois são setados utilizando as classes
YansWifiChannelHelper e YansWifiPhyHelper, respectivamente.

O canal comum em todos os dispositivos garante que todos os dispositivos possam se comunicar e interferir uns com os
outros.

Uma vez que a camada física da rede está configurada, podemos então focar na camada MAC. Para esta camada devemos
definir um helper, no qual indicamos o SSID da rede e outras configurações. Para estações e pontos de acesso utilizam-se
diferentes camadas MAC.

O objeto wifiHelper é responsável por instalar os dispositivos wifi nos nós. Nele também é possível alterar o padrão de
wifi e especificar outros atributos. Para que este objeto consiga instalar os dispositivos corretamente utiliza-se a
camada física, a camada MAC e o container de nós como argumentos. O helper também é responsável por definir o
gerenciador de rede, esta escolha pode influenciar bastante a depender do cenário.

Assim como outros dispositivos de rede, os dispositivos retornados pela função Install do wifiHelper são guardadas em um
NetDeviceContainer.

\section{Mobilidade}

Utiliza-se o MobilityHelper para configurar o posicionamento dos aparelhos. A função SetPositionAllocator organiza, por
meio de atributos o posicionamento dos objetos.

Após alocar as posições iniciais dos aparelhos, utiliza-se a função SetMobilityModel para adicionar modelos aleatórios
ou determinísticos para a variação da posição ao longo do tempo. Há ainda o modelo ns3::ConstantPositionMobilityModel
para pontos de acesso fixos.

O objeto criado de MobilityHelper deve ser então instalado nos nós móveis da rede.

\section{TCP/IP}

Para instalar a pilha de protocolos TCP/IP utiliza-se a classe InternetStackHelper. Após isso será necessário configurar
o endereço IP dos nós. O endereço pode ser IPv6 ou IPv4, sendo Ipv6AddressHelper e Ipv4AddressHelper suas funções
auxiliares. Nestas é necessário informar o endereço da rede e uma máscara.

Após especificados os endereços utiliza-se a função Assign para atribuí-los a nós. Recomenda-se guardar as interfaces
criadar em um container Ipv6InterfaceContainer/Ipv4InterfaceContainer.

Se utilizarmos diferentes tecnologias de comunicação e em IPv4, é necessário popular a tabelas de roteamento. Isso é realizado
pela classe Ipv4GlobalRoutingHelper com a função PopulateRoutingTables.

\subsection{Cuidados}

Em uma rede Wi-Fi o ponto de acesso gera feixes de sinais periodicamente, portanto não há um momento de término direto
da simulação. Diversos eventos são agendados indefinidamente pelo ponto de acesso. Assim, é necessário especificar um
ponto de parada da simulação através do Simulator::Stop().

A escolha de filas e suas políticas de gerenciamento tem um grande impacto na performance da rede. Duas camadas de filas
são utilizadas para emular as filas implementadas em software e hardware. O ajuste dinâmico do tamanho das filas pode
ser feito ativando o algoritmo BQL (Byte Queue Limits). Tipicamente as filas podem ser alteradas no helper que define os
dispositivos de rede e um helper de controle de tráfego, TrafficControlHelper, pode ser utilizado para alterar atributos
desta fila e instalados nos dispositivos.





\end{document}

