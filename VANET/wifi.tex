\documentclass{article}
\usepackage[brazilian]{babel}
\usepackage[utf8]{inputenc}
\usepackage[T1]{fontenc}
\usepackage{listings}

\author{Renê Cardozo \\ 
        rene.cardozo@usp.br}

\title{Módulo Wi-Fi}

\date{}

\begin{document}
\maketitle

O dispositivo de rede Wi-Fi, WifiNetDevice pode ser instalado em um nó para modelar um controlador Wi-Fi. Cada nó pode
ter mais de um dispositivo de rede WiFi em diferentes canais e podem estar instalados junto de outros dispositivos
relacionados a outras tecnologias de comunicação.

\section{Camadas}

\subsection{Camada MAC}
Existem três tipos de modelo de alto nível para a camada MAC da rede Wi-Fi, os quais são aplicados pela função setType
do wifiMacHelper. São estes: ns3::ApWifiMac, ns3::StaWifiMac e ns3::AdhocWifiMac. 

\subsection{Camada Física}

A camada física é responsável por modelar a recepção de pacotes e manter registro do consumo de energia.

\end{document}

