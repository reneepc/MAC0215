\documentclass{article}
\usepackage[brazilian]{babel}
\usepackage[utf8]{inputenc}
\usepackage[T1]{fontenc}
\usepackage{listings}

\author{Renê Cardozo \\ 
        rene.cardozo@usp.br}

\title{Módulo de Internet}

\date{}

\begin{document}
\maketitle

\section{Camadas}

\subsection{Camada 3}
Na camada mais baixa, acima dos dispositivos de rede, temos os protocolos IPv4, IPv6, ARP e outros. A classe
Ipv4L3Protocol é um exemplo da interface pública da classe Ipv4.

\subsection{Camada 4}

Esta camada especifica os protocolos de transporte, sockets e aplicações. Toda implementação de um protocolo de
transporte é uma fábrica de sockets. Por exemplo, para criar um socket UDP:

\begin{lstlisting}
Ptr<Udp> udpSocketFactory = GetNode()->GetObject<Udp>();
Ptr<Socket> m\_socket = socketFactory->CreateSocket();
m\_socket->Bind(m\_local\_address);
...
\end{lstlisting}

\section{TCP}
O uso do TCP é geralmente definido na camada de aplicação, informando o tipo de fábrica de socket que deve ser
utilizada.
\end{document}

